%%%%%%%%%%%%%%%%%%%%%%%%%%%%%%%%%%%%%%%%%%%%%%%%%%%%%%%%%%%%%%%%%%%%%%
% How to use writeLaTeX: 
%
% You edit the source code here on the left, and the preview on the
% right shows you the result within a few seconds.
%
% Bookmark this page and share the URL with your co-authors. They can
% edit at the same time!
%
% You can upload figures, bibliographies, custom classes and
% styles using the files menu.
%
%%%%%%%%%%%%%%%%%%%%%%%%%%%%%%%%%%%%%%%%%%%%%%%%%%%%%%%%%%%%%%%%%%%%%%

\documentclass[12pt]{article}

\usepackage{sbc-template}
\usepackage{xcolor}
\usepackage{graphicx,url}
\usepackage{amsmath, amsfonts, amssymb}

%\usepackage[brazil]{babel}   
\usepackage[utf8]{inputenc}  

\newcommand{\helena}[1]{\textcolor{magenta}{\textbf{HELENA:} #1}}
\newcommand{\bruno}[1]{\textcolor{blue}{\textbf{BRUNO:} #1}}
\sloppy

\title{Investigações acerca da modalidade da necessidade}

\author{Bruno Rafael dos Santos\inst{1}, Elian Gustavo Chorny Babireski\inst{1},\\Helena Vargas Tannuri\inst{1}, Vinícios Bidin dos Santos\inst{1}}


\address{Departamento de Ciência da Computação {--} Universidade do Estado de Santa Catarina
  (UDESC)\\
  Joinville {--} SC {--} Brazil
  \email{\{bruniculos2014, elian.babireski, helenavargastannuri, vinibidin\}@gmail.com}
}

\begin{document} 

\maketitle

\begin{abstract}
  This meta-paper describes the style to be used in articles and short papers
  for SBC conferences. For papers in English, you should add just an abstract
  while for the papers in Portuguese, we also ask for an abstract in
  Portuguese (``resumo''). In both cases, abstracts should not have more than
  10 lines and must be in the first page of the paper.
\end{abstract}
     
\begin{resumo} 
  Este meta-artigo descreve o estilo a ser usado na confecção de artigos e
  resumos de artigos para publicação nos anais das conferências organizadas
  pela SBC\@. É solicitada a escrita de resumo e abstract apenas para os artigos
  escritos em português. Artigos em inglês deverão apresentar apenas abstract.
  Nos dois casos, o autor deve tomar cuidado para que o resumo (e o abstract)
  não ultrapassem 10 linhas cada, sendo que ambos devem estar na primeira
  página do artigo.
\end{resumo}


\section{Introdução}
% (1) Explicar o que é necessidade
A lógica modal é o estudo das relações de consequências sobre proposições necessariamente verdadeiras.
A necessidade é um conceito tratado na lógica modal alética, a qual por sua vez é o dual de possibilidade, e portanto são interdefiníveis. Sendo $\square P$ a necessidade da proposição $P$ e $\Diamond P$ a possibilidade de $P$, então, $\Diamond P \equiv \neg \square \neg P$, analogamente, $\square P \equiv \neg \Diamond \neg P$. Estes conceitos são conhecidos como modalidades.

% (2) Aristóteles
Aristóteles, em seu trabalho~\cite{Aristotle-A}, acopla os conceitos de modalidade ao seu sistema de silogismo dedutivo. \helena{escrever mais sobre isso}
% (2) Lewis


\section{Definição da lógica modal alética} \label{sec:definicao}
  \helena{blablablablalblla \cite{silveira2020implementacao}}
  \subsection{Linguagem} \label{sec:linguagem}
  \helena{EU QUERO FAZER ISSO AQUI}
  \subsection{Sistema dedutivo} \label{sec:dedutivo}
  \bruno{VOU FAZER ISSO O QUE VCS ACHAM?}

\subsection{Semântica dos mundos possíveis} \label{sec:semantica}


\section{Aplicações na computação} \label{sec:aplicacoe}


\section{References}


\bibliographystyle{sbc}
\bibliography{sbc-template}

\end{document}
